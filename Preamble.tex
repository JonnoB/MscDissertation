\maketitle
\makedeclaration

\begin{abstract} % 300 word limit

This project takes a novel approach to 24 hour ahead domestic electricity forecasting of the evening peak demand by classifying smart meter consumption behaviours into groups using unsupervised cluster detection on a graph. A transition matrix is used to calculate the probability of transferring between clusters, a 1 step ahead markov chain is then used to predict the total number of smart meters in each cluster the following day. Finally the nodes are used to scale the mean cluster profile of the previous day to create a profile forecast. The model makes it possible for electricity suppliers to more accurately buy production on the day ahead market and so reduce reliance on the spot market to cover deficit's or over production. This in turn reduces the cost of electricity for the consumer and the provider and ultimately reducing the requirement for spinning reserve thus $CO_2$ production. 7 clusters were detected plus the Node Soup, with clusters defined by the time of the main peak. The clusters are found using the Louvain clustering algorithm which greedily optimises the modularity of the network. The model had a MAPE of 3.9\%, competitive with literature benchmarks and outperforming a simple linear model which had a MAPE of 5\%.

The same technique is used to predict what behaviour type each smart meter may exhibit the following day. The model has an accuracy of 25\% across 8 clusters, an XGboost model had an accuracy of 36\%. Although less successful than the regression method, does point a way for electricity providers to help vulnerable customers avoid high charges by sending reminders if their behaviour indicates they may have high use during peak hours. 

Finally a classifier was developed in an attempt to predict social demographic class based on cluster behaviour. This model was not successful performing no better than chance. This is believed to be down to two reasons, primarily the low levels of predictable behaviour for each smart meter, and secondly the reliability of using MOSAIC demographic profiling as the dependent variable. 

\end{abstract}

\begin{acknowledgements}
Thank you to Chris Lowery at Baringa Partners for introducing me to this dataset. Dr Serguieva for her support on the dissertation writing process. My Mum for her patience in reviewing the work. And finally Dr Shipworth and Dr O'Sullivan for their invaluable guidance and feedback throughout the dissertation on both understanding the UK energy sector and also analytical insight, I am lucky to have such engaged and supportive supervisors. 
\end{acknowledgements}

\setcounter{tocdepth}{1} 
% Setting this higher means you get contents entries for
%  more minor section headers.

\tableofcontents
\listoffigures
\listoftables

